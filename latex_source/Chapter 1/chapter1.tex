\section {Specifiche informali}

Si intende sviluppare un sistema software per la gestione di un servizio di carpooling destinato a facilitare la condivisione di viaggi per pendolari. 

Questo sistema consentirà agli utenti di offrire posti disponibili nelle loro auto o di trovare un passaggio per le loro trasferte quotidiane o occasionali.
Gli utenti potranno registrarsi al sistema inserendo dati personali come nome, cognome, contatto telefonico, indirizzo email, tipo di automobile, numero di posti disponibili. Dopo la registrazione, ogni utente potrà pubblicare i dettagli dei viaggi che intende condividere, includendo partenza, destinazione, data e ora di partenza, e contributo richiesto per le spese di viaggio.
Il sistema permetterà agli utenti registrati di cercare un passaggio inserendo i loro criteri di ricerca ossia punto di partenza, destinazione e data. I risultati mostreranno le corrispondenze disponibili. Gli utenti potranno prenotare un posto direttamente attraverso la piattaforma. La prenotazione sarà un biglietto che contiene un ID, riferimento del guidatore, riferimento del passeggero e costo del viaggio.

Per gli autisti, il sistema fornirà una interfaccia grafica che mostra l’elenco delle sue prenotazioni con la possibilità di accettarle oppure eventualmente di poterle cancellare. Sarà anche possibile per gli autisti valutare i passeggeri a fine viaggio, contribuendo così a mantenere un ambiente di viaggio sicuro e piacevole per tutti.
Il sistema includerà anche una funzionalità di feedback reciproco tra autisti e passeggeri per mantenere un alto standard di fiducia e sicurezza nella community. 

Il gestore dell’applicazione invece, tramite una opportuna interfaccia grafica può accedere a delle funzioni di reportistica, che permetto di visualizzare l’incasso totale, elenco di passeggeri e guidatori del sistema, con relativa valutazione.